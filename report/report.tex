\documentclass[12pt,oneside,a4paper]{article}

\usepackage[backend=biber,style=numeric]{biblatex}
\usepackage{xcolor}
\usepackage{todonotes}
\usepackage{amsmath}
\usepackage{multicol}
\usepackage{caption}
\usepackage{hyperref}
\usepackage{graphicx}
\usepackage{listings}
\lstset{
	frame=top,frame=bottom,
	language=C,
	basicstyle=\small\normalfont,
	xleftmargin=\parindent,
	keywordstyle=\color{green!40!black},
	%  commentstyle=\itshape\color{purple!40!black},
	%  identifierstyle=\color{blue},
	%  stringstyle=\color{orange},
	morekeywords={in, globaldata, procedure, input, output, behavior, end, XOR, NOT, AND}, % keyword to highlight
	%  captionpos=t,
	tabsize=2,
	numbers=left,
	stepnumber=1,                   % the step between two line-numbers.        
	numbersep=5pt,
	framexleftmargin=10pt,
	title=\lstname,
	captionpos=t,
	showspaces=false,
}
\DeclareCaptionFormat{listing}{\rule{\dimexpr\textwidth\relax}{0.4pt}\par\vskip1pt#1#2#3}
\captionsetup[lstlisting]{format=listing,singlelinecheck=false, margin=0pt,labelsep=space,labelfont=bf}

\usepackage{booktabs}
\usepackage[noabbrev,capitalise]{cleveref}
\crefname{listing}{algorithm}{algorithms}
\Crefname{listing}{Algorithm}{Algorithms}
\renewcommand\lstlistingname{Algorithm}
\def\lstlistingcrefname{Algorithm}
\usepackage{url}

\addbibresource{biblio.bib}

\title{\textbf{Example title}}

\author{Simulation of Quantum Circuits on FPGAs\\Federico Lolli, Angelo Zangari}

\date{\today}

\begin{document}

\begin{titlepage}
    \centering
    \clearpage
    \maketitle
	\thispagestyle{empty}
	\vspace*{1cm}
	\includegraphics[width=4cm]{example.jpg} % qui mettete il vostro logo, o cancellate la linea
	\vfill
	\centering
    % FIXME: choose 1 of the 2 footers below
	\includegraphics{footer.png}
	\includegraphics{logo_polimi.png}\includegraphics{logo_NECST.png}
\end{titlepage}

\begin{abstract}
%    Enter a short summary here. What topic do you want to investigate and why? What experiment did you perform? What were your main results and conclusion?
// DESCR intro + problem description
In the modern era of quantum computing, the need for simulation of quantum circuits on classical computers has become of paramount importance. When talking about classical computers running quantum circuits, mainly we are referring to two tasks: simulation and verification. Both of these techniques used to rely on a state vector based representation. The limitation of this approach is that the state vector dimension grows exponentially with respect to the circuit size, which renders this method spatially infeasible for larger circuits. To solve this issue we can resort to tensor network based algorithms, which map all components of the quantum circuit to appropriate tensors, forming a tensor network that can then be contracted to obtain the final state bitstrings amplitudes.

\end{abstract}

\section{Introduction} 

\subsection{Objective and Scope}
% State the purpose of your project, the objectives, and the scope. Explain why developing a quantum simulation accelerator is important.

\subsection{Significance}
% Discuss the significance of this work in the context of current technological advancements and potential applications.

\subsection{Overview}
% Provide a brief overview of the report structure.


\section{Problem Explanation (Quantum Simulation)} 

\subsection{Background}
% Explain what quantum simulation is and why it is challenging. Include the theoretical foundations and the practical implications.

\subsection{Current Challenges}
% Highlight the key challenges in quantum simulation that your project aims to address.

\subsection{Motivation}
% Explain why you chose to tackle this problem and how it relates to existing research or industry needs.


\section{State of the Art Reference} 

\subsection{Literature Review}
% Summarize existing work in the field of quantum simulation accelerators. Discuss various approaches, technologies used, and their limitations.

\subsection{Comparison}
% Compare different methods and highlight where your ap- proach fits in the current landscape.

\subsection{Gaps}
% Identify the gaps in the current state of the art that your project aims to fill.



\section{Methods} 

\subsection{Design Approach}
% Describe the overall design and methodology used to develop the quantum simulation accelerator.

\subsection{Technology Stack}
% Provide an overview of the technologies and tools used, including Vitis HLS, Rust, and OpenCL.

\subsection{Development Process}
% Outline the steps taken in the development process, from initial design to final implementation.


\section{Frontend (Host Side) Explanation} 

\subsection{Architecture}
% Explain the architecture of the host side implemented in Rust. Include diagrams if necessary.

\subsection{Implementation Details}
% Provide detailed descriptions of the key com- ponents and their functionality.

\subsection{Communication}
% Explain how the host side communicates with the kernel using OpenCL. Detail any challenges faced and how they were overcome.

\subsection{Code Examples}
% Include snippets of important sections of your Rust code, explaining what each part does.



\section{Kernel (FPGA Side) Explanation} 

\subsection{Architecture}
% Describe the FPGA kernel’s architecture and how it accelerates quantum simulations.

\subsection{Implementation Details}
% Provide detailed explanations of the implementation using Vitis HLS. Discuss any optimizations made for performance.

\subsection{Integration}
% Explain how the kernel integrates with the host side via OpenCL. Include diagrams and flowcharts to illustrate the process.

\subsection{Code Examples}
% Include HLS code snippets and explain their functionality.


\section{Results and Discussion} 

\subsection{Performance Metrics}
% Present the performance metrics of your ac- celerator. Compare them against benchmarks or other state-of-the-art solutions.

\subsection{Analysis}
% Analyze the results, discussing any improvements or unexpected outcomes. Use charts and graphs to visualize data.

\subsection{Limitations}
% Discuss the limitations of your current implementation and any potential improvements for future work.


\section{Conclusion} 

\subsection{Summary}
% Summarize the key findings and contributions of your project.

\subsection{Future Work}
% Suggest areas for future research or improvements that can be made to your quantum simulation accelerator.

\subsection{Final Thoughts}
% Provide any concluding remarks or reflections on the project.


\section{References} 

\subsection{Citations}
% List all the references cited in your report in the appropriate academic format.


\section{Appendices}
% Include any supplementary material, such as full code listings or additional data, in the appendices.


% Additional Tips 
    % Figures and Charts: Ensure that all figures, charts, and diagrams are clearly labeled and referenced in the text. Use them to illustrate complex points and provide visual clarity.
    % Clarity and Conciseness: Make sure your writing is clear and concise. Avoid unnecessary jargon and explain technical terms where needed.


\end{document}