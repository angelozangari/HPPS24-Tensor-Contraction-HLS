\documentclass[12pt,oneside,a4paper]{article}

\usepackage[backend=biber,style=numeric]{biblatex}
\usepackage{xcolor}
\usepackage{todonotes}
\usepackage{amsmath}
\usepackage{multicol}
\usepackage{caption}
\usepackage{hyperref}
\usepackage{graphicx}
\usepackage{listings}
\lstset{
	frame=top,frame=bottom,
	language=C,
	basicstyle=\small\normalfont,
	xleftmargin=\parindent,
	keywordstyle=\color{green!40!black},
	%  commentstyle=\itshape\color{purple!40!black},
	%  identifierstyle=\color{blue},
	%  stringstyle=\color{orange},
	morekeywords={in, globaldata, procedure, input, output, behavior, end, XOR, NOT, AND}, % keyword to highlight
	%  captionpos=t,
	tabsize=2,
	numbers=left,
	stepnumber=1,                   % the step between two line-numbers.        
	numbersep=5pt,
	framexleftmargin=10pt,
	title=\lstname,
	captionpos=t,
	showspaces=false,
}
\DeclareCaptionFormat{listing}{\rule{\dimexpr\textwidth\relax}{0.4pt}\par\vskip1pt#1#2#3}
\captionsetup[lstlisting]{format=listing,singlelinecheck=false, margin=0pt,labelsep=space,labelfont=bf}

\usepackage{booktabs}
\usepackage[noabbrev,capitalise]{cleveref}
\crefname{listing}{algorithm}{algorithms}
\Crefname{listing}{Algorithm}{Algorithms}
\renewcommand\lstlistingname{Algorithm}
\def\lstlistingcrefname{Algorithm}
\usepackage{url}

\addbibresource{biblio.bib}

\title{\textbf{Example title}}

\author{Group Name\\Author 1, Author 2}

\date{\today}

\begin{document}


\begin{titlepage}
	\centering
	\clearpage
	\maketitle
	\thispagestyle{empty}
	\vspace*{1cm}
	\includegraphics[width=4cm]{example.jpg} % qui mettete il vostro logo, o cancellate la linea
	\vfill
	\centering
	\includegraphics{logo_polimi.png}\includegraphics{logo_NECST.png}
\end{titlepage}


\begin{abstract}
Enter a short summary here. What topic do you want to investigate and why? What experiment did you perform? What were your main results and conclusion?
\end{abstract}

\section{Introduction} \label{sec:intro}
Please, be careful about paragraph separation: double backslash (\textbackslash\textbackslash) is used to break the line without spacing, and should be used only to break a single, long paragraph. Instead, to create a new paragraph two line breaks should be used.

This is, for example, a new, properly spaced paragraph (you can see the indentation at the beginning). Remember you can put references to external sources, in particular scientific articles or websites. To cite a source, you need to add the reference to the ``biblio.bib'' file in the Bibtex format, which lists the main information about the source. Since writing a Bibtex reference manually can be long, you can usually find the whole Bibtex reference on the internet, for example in the IEEE or ACM websites. Google is always a good source. In general, the procedure to cite a new source is:
\begin{enumerate}
	\item find the Bibtex reference (if any)
	\item copy or manually write it into the ``biblio.bib'' file
	\item give it a label you like (in this example, you can see ``vantage'' or ``sandy\_slides''), provided it is \emph{unique}
	\item use it throughout the text, with the \textbackslash \emph{cite} directive: for example, \cite{vantage} or \cite{sandy_slides}
\end{enumerate}
The list of referenced sources will appear at the end of the report.

\section{Section 1} \label{sec:sec1}
In the following, \cref{eq:exeq} shows an example of equation centered within the page.

\begin{equation}\label{eq:exeq}
%
\mbox{maximize} \sum_{i=17}^{31} \sum_{j=i+1}^{32} [ x_{i,j} \times s_{i,j} + (1 - x_{i,j}) \times d_{i,j} ]
%
\end{equation}

To type any mathematical expression in the text without breaking the line, you can surround it with the \$ symbol, for example to refer to $i$ and to $ \sum_{j=i+1}^{32} [ x_{i,j} \times s_{i,j} + (1 - x_{i,j}) \times d_{i,j} ] $.

If you need to show code snippets, you can use the \emph{listing} environment, as in the following example. As for the other elements, you can refer to a listing through its label as in \cref{list:exlistings}. Remember to make your code well readable by indenting it and using concise pseudo-code snippets, without pasting your own code as it is (unless it is REALLY expressive and short).

\begin{lstlisting}[label={list:exlistings},caption={Example of code snippet}]
globaldata: list_head buddies[MAX_ORDER][MAX_COLORS]

procedure InsertBuddy(buddy b, order ord)
	buddy twin
	mcolor mcol
	
	mcol = Mcolor(b, ord)
	twin = GetTwinBuddy(b,ord)	
	if ord < MAX_ORDER-1 AND BuddyIsFree(twin)
		RemoveFromList(buddies[ord][Mcolor(twin,ord)])
		b = CoalesceBuddy(b, twin, ord)
		InsertBuddy(b, ord+1)
		return
	else
		InsertHead(buddies[ord][mcol],b)
end procedure
\end{lstlisting}

\subsection{Subsection 1} \label{sec:sub1}
This is the way to refer to \cref{fig:example}, and similarly for \cref{sec:sec1}.
\begin{figure}
	\centering
	\includegraphics[width=.7\textwidth]{example.jpg}
	\caption{Example caption.}
	\label{fig:example}
\end{figure}
You will notice \LaTeX~ freely moves elements like figures and tables around the page, and often in the pages around the current paragraph. In particular, \LaTeX~ always places these elements at the bottom or top of the page (otherwise instructed): this choice obeys to the main typesetting guidelines, and should work well most of the times. You should not force a specific position for these elements, and keep in mind that \textit{\LaTeX~ most of the time is right} (it is its job to do lay out elements, not yours). If you need to move an element, move its \LaTeX~ code up or down.

\subsection{Subsection 2} \label{sec:sub2}
\Cref{tab:example} provides an example of a table. According to many people, this table style (without vertical lines separating columns) is the most elegant and clean possible; to set this tables style, this document adds the \verb|\usepackage{booktabs}| directive at the beginning. In the \LaTeX~ code, you can notice that an ampersand (\$) separates columns and a double backslash (\verb|\\|) moves to a new line.
\begin{table}
	\caption{Table title (without stop!)}
	\centering
	\begin{tabular}{ccl}
		\toprule
		\textbf{label0} & \textbf{label1} & \textbf{label2} \\
		\midrule
		row 0, col 0 & row 0, col 1 & row 0, col 2 \\
		row 1, col 0 & row 1, col 1 & row 1, col 2 \\ 
		row 2, col 0 & row 2, col 1 & row 2, col 2 \\ 
		row 3, col 0 & row 3, col 1 & row 3, col 2 \\
		\bottomrule
	\end{tabular}
	\label{tab:example}
\end{table}

Since tables in \LaTeX~ are verbose, you should:
\begin{itemize}
\item place them in a specific file, to be included with a \verb|\input{filename}| directive
\item for large tables, fill them on applications or websites like \url{https://www.tablesgenerator.com/}, then copy their code and paste it in the dedicated file; finally, you can customize the style from the \LaTeX~ code
\end{itemize}

Here you can see the same table as before but included from an external file \textit{table.tex}: the result is the same.
%
\begin{table}
	\caption{Table title (without stop!)}
	\centering
	\begin{tabular}{ccl}
		\toprule
		\textbf{label0} & \textbf{label1} & \textbf{label2} \\
		\midrule
		row 0, col 0 & row 0, col 1 & row 0, col 2 \\
		row 1, col 0 & row 1, col 1 & row 1, col 2 \\ 
		row 2, col 0 & row 2, col 1 & row 2, col 2 \\ 
		row 3, col 0 & row 3, col 1 & row 3, col 2 \\
		\bottomrule
	\end{tabular}
	\label{tab:example2}
\end{table}


\printbibliography

\end{document}


